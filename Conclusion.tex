\chapter{Conclusion}

As far as we know, these are the first peridynamic state-based thin feature models.

The models developed so far provide allow for peridynamic modeling of peridynamic beams and plates in elastic bending.
The models are validated by comparing their strain energies to the strain energy of classical models for small, homogenous deformations.
Code was written to evaluate both beam and plate bond-pair models for linear elastic, brittle, nonlinear elastic, and elastic perfectly plastic materials.
Simulations run with the developed models provide results in agreement with conventional methods for simple cantilever beam tests.
Plate simulation results are promising but not yet verified for either elastic or elastic perfectly plastic materials.
The proposed damage model successfully reproduces the impact of nonlinear elasticity on deformation of a rectangular cantilever, and the framework is laid to allow application of the same model to plates and I-beams.
Plastic deformation results for rectangular cantilevers are promising, but the residual deformation for a plastically deformed beam does not yet match expected results.

There are still several several goals that will improve the reliability, applicability, and usability of these models.
While a few cantilever beam cases have been validated, extended validation will consider fixed-displacement (pinned) boundary conditions and additional load cases.
Correctly modeling each case will require a more careful consideration of the appropriate way to apply various loads and enforce various boundary conditions.
Further testing will also include brittle failure and plasticity in I-beams and in plates.
Specifically, the hysteresis associated with plastic deformation must be reconciled.

To maximize computational efficiency, the presented results were obtained by fixing the in-plane coordinates of each node and allowing only the vertical component to vary.
The presented cases are expected to exhibit almost no lateral displacements, but to generalize the model it will need to include all three degrees of freedom for each node.
The code to accomplish this task was written and commented out to improve performance, so this should be a relatively easy change.
The increased computational complexity may require implementation of a different solver method such the Nonlinear Conjugate Gradient Method.
Neither the bond-pair nor the bond-multiple models resist bond extension, so full degree-of-freedom models must also incorporate an extension energy term to prevent infinite transverse displacements.

A full degree-of-freedom model is necessary to extend the plate model for use with 3D thin shells.
Shells are similar to plates but are not initially a flat plane.
It is expected that shells can be successfully by starting with a plate and applying initial plastic deformations to all bond pairs that are not initially collinear.

By incorporating extension energy into the model, we will also gain the ability to model buckling and tension-stiffening, both of which are important to thin feature failure.
Beam, plate, and shell elements should all exhibit buckling and tension stiffening behavior as a natural result of combining models that resist bending with those that resist elongation.

To make these models more useful for analysis of real parts, they should be extended to be usable with irregular discretization.
For beams, this may be accomplished by using the bond-multiple model.
For plates, irregular discretization might be possible using a modified bond-multiple method, or it may require the implementation of virtual points at the edges of regions with finer discretization.

Finally, the fact that the bond-pair plate model describes a material with the same Poisson ratio as a bond-pair 2D peridynamic solid suggests a means of modeling plates with arbitrary Poisson ratios.
Just as the state-based linear peridynamic solid divides the deformation state into dilatory and deviatoric deformation, it may be possible to divide the bending state of a plate into spherical and deviatoric bending.



\appendix
\chapter{Fr\'echet Derivative}
\label{sec:frechet}
\section{Definition}
The derivative of a function of a state is defined by Silling in \cite{silling2007peridynamic} as follows:
\begin{quote}
Let $\Psi$ be a function of a state, $\Psi(\cdot):\mathcal{A}_m\rightarrow\mathcal{L}_n$. Suppose there exists a state-valued function denoted $\nabla\Psi\in\mathcal{A}_{m+n}$ such that for any $\vstate{A}{}{}\in\mathcal{A}_m$ and any $\Delta\vstate{A}{}{}\in\mathcal{A}_m$,
\begin{equation}
  \Psi(\vstate{A}{}{}+\Delta\vstate{A}{}{})=\Psi(\vstate{A}{}{})+\nabla\Psi(\vstate{A}{}{})\bullet\Delta\vstate{A}{}{}+o(||\Delta\vstate{A}{}{}||).
\end{equation}
Then $\Psi$ is said to be \textit{differentiable} and $\nabla\Psi$ is called the \textit{Frechet derivative} of $\Psi$.
\end{quote}
This is a fairly straightforward way of defining a derivative with respect to a state.
Because the force vector-state and deformation vector-state are work conjugate, the force vector-state can be determined by taking the Fr\'echet derivative of energy with respect to the deformation vector-state.
\section{Bond-Pair Force}
For the bond-pair model, we derive the bond force function from the bond-pair energy function
%
\begin{align}
%
\tvstate{T}{}{\boldsymbol{\xi}} &= \nabla w\!\left(\tvstate{Y}{}{\boldsymbol{\xi}}\right)\notag \\
%
w&=\omega\!\left(\boldsymbol{\xi}\right)\alpha\left[1+\cos\!\left(\theta\!\left(\tvstate{Y}{}{\boldsymbol{\xi}},\tvstate{Y}{}{-\boldsymbol{\xi}}\right)\right)\right]\notag \\
%
w\!\left(\tvstate{Y}{}{\boldsymbol{\xi}}+\Delta\tvstate{Y}{}{\boldsymbol{\xi}}\right) &=
 \omega\!\left(\boldsymbol{\xi}\right)\alpha\left[1+\cos\!\left(\theta\!\left(\tvstate{Y}{}{\boldsymbol{\xi}}+\Delta\tvstate{Y}{}{\boldsymbol{\xi}},\tvstate{Y}{}{-\boldsymbol{\xi}}\right)\right)\right]\notag
\end{align}
%
\begin{multline}
\nabla w\!\left(\tvstate{Y}{}{\boldsymbol{\xi}}\right)\bullet\Delta\tvstate{Y}{}{\boldsymbol{\xi}}= 
w\!\left(\tvstate{Y}{}{\boldsymbol{\xi}}+\Delta\tvstate{Y}{}{\boldsymbol{\xi}}\right) -w\!\left(\tvstate{Y}{}{\boldsymbol{\xi}}\right) \notag\\
%
=\omega\!\left(\boldsymbol{\xi}\right)\alpha\sin\!\left(\theta\!\left(\tvstate{Y}{}{\boldsymbol{\xi}},\tvstate{Y}{}{-\boldsymbol{\xi}}\right)\right)\!\left[\theta\!\left(\tvstate{Y}{}{\boldsymbol{\xi}}+\Delta\tvstate{Y}{}{\boldsymbol{\xi}},\tvstate{Y}{}{-\boldsymbol{\xi}}\right)-\theta\!\left(\tvstate{Y}{}{\boldsymbol{\xi}},\tvstate{Y}{}{-\boldsymbol{\xi}}\right)\right]
\end{multline}
%
\begin{equation}
\left[\theta\!\left(\tvstate{Y}{}{\boldsymbol{\xi}}+\Delta\tvstate{Y}{}{\boldsymbol{\xi}},\tvstate{Y}{}{-\boldsymbol{\xi}}\right)-\theta\!\left(\tvstate{Y}{}{\boldsymbol{\xi}},\tvstate{Y}{}{-\boldsymbol{\xi}}\right)\right] = 
\frac{\Delta\tvstate{Y}{}{\boldsymbol{\xi}}}{|\tvstate{Y}{}{\boldsymbol{\xi}}|}\bullet \hat{\theta}\!\left(\tvstate{Y}{}{\boldsymbol{\xi}},\tvstate{Y}{}{-\boldsymbol{\xi}}\right)\notag
\end{equation}
%
To determine the $\hat{\theta}$ direction vector, we must construct a vector that is normal to $\tvstate{Y}{}{\boldsymbol{\xi}}$ and that is in the plane containing both $\vstate{Y}{}{\boldsymbol{\xi}}$ and $\vstate{Y}{}{\boldsymbol{-\xi}}$.
The cross product of $\vstate{Y}{}{\boldsymbol{\xi}}$ and $\vstate{Y}{}{\boldsymbol{-\xi}}$ is a vector normal to that plane, so any vector normal to that cross product will be in the correct plane.
Therefore, the vector $\vstate{Y}{}{\boldsymbol{\xi}}\times\left[\vstate{Y}{}{\boldsymbol{\xi}}\times\vstate{Y}{}{\boldsymbol{-\xi}}\right]$ is both normal to $\vstate{Y}{}{\boldsymbol{\xi}}$ and is in the plane containing both $\vstate{Y}{}{\boldsymbol{\xi}}$ and $\vstate{Y}{}{\boldsymbol{-\xi}}$.
Normalizing gives us the $\hat{\theta}$ direction vector:
%
\begin{equation}
\hat{\theta}\!\left(\tvstate{Y}{}{\boldsymbol{\xi}},\tvstate{Y}{}{-\boldsymbol{\xi}}\right)=
\frac{\tvstate{Y}{}{\boldsymbol{\xi}}\times\left[\tvstate{Y}{}{\boldsymbol{\xi}}\times\tvstate{Y}{}{\boldsymbol{-\xi}}\right]}{|\tvstate{Y}{}{\boldsymbol{\xi}}||\tvstate{Y}{}{\boldsymbol{\xi}}||\tvstate{Y}{}{\boldsymbol{-\xi}}|\sin\!\left(\theta\!\left(\tvstate{Y}{}{\boldsymbol{\xi}},\tvstate{Y}{}{-\boldsymbol{\xi}}\right)\right)}\notag 
\end{equation}
%
We combine all of these to get the expression for bond force found in \cref{eq:SillingForceNO}.
\begin{align}
%
\tvstate{T}{}{\boldsymbol{\xi}} &=
\omega\!\left(\boldsymbol{\xi}\right)\frac{-\alpha}{|\tvstate{Y}{}{\boldsymbol{\xi}}|} \frac{\tvstate{Y}{}{\boldsymbol{\xi}}}{|\tvstate{Y}{}{\boldsymbol{\xi}}|} \times 
\left[\frac{\tvstate{Y}{}{\boldsymbol{\xi}}}{|\tvstate{Y}{}{\boldsymbol{\xi}}|} \times 
\frac{\tvstate{Y}{}{-\boldsymbol{\xi}}}{|\tvstate{Y}{}{-\boldsymbol{\xi}}|}\right]\notag
\end{align}
%
\section{Isotropic Bending Correction}
To derive the bending ``pressure'' force, we start with the isotropic energy discrepancy
%
\begin{equation}
    W'=2G\frac{h^3}{12}\frac{3\nu-1}{1-\nu}\bar{\boldsymbol{\kappa}}^2. \notag
\end{equation}
%
with
%
\begin{align}
    \bar{\boldsymbol{\kappa}}\left(\vstate{Y}{}{}\right) &= \frac{1}{m} \int_0^\delta \int_0^{2\pi}\omega(\xi)\frac{\vstate{Y}{}{\boldsymbol{\xi}}+\vstate{Y}{}{\boldsymbol{-\xi}}}{\xi^2} \xi {\rm d}\phi {\rm d}\xi \notag \\
    &= \frac{2}{m} \int_0^\delta \int_0^{2\pi}\omega(\xi)\frac{\vstate{Y}{}{\boldsymbol{\xi}}}{\xi^2} \xi {\rm d}\phi {\rm d}\xi \notag
\end{align}
Because $\bar{\boldsymbol{\kappa}}$ is itself a vector-state, we will need to begin with the change in $\bar{\boldsymbol{\kappa}}$ with respect to $\vstate{Y}{}{}$ and carry the result through to find the change in $W'$.
\begin{align}
    \bar{\boldsymbol{\kappa}}\left(\vstate{Y}{}{}+\Delta\vstate{Y}{}{}\right) &= \frac{2}{m} \int_0^\delta \int_0^{2\pi}\omega(\xi)\frac{\vstate{Y}{}{\boldsymbol{\xi}}+\Delta\vstate{Y}{}{\boldsymbol{\xi}}}{\xi^2} \xi {\rm d}\phi {\rm d}\xi \notag \\
    &= \bar{\boldsymbol{\kappa}}\left(\vstate{Y}{}{}\right) + \frac{2}{m} \int_0^\delta \int_0^{2\pi}\omega(\xi)\frac{\Delta\vstate{Y}{}{\boldsymbol{\xi}}}{\xi^2} \xi {\rm d}\phi {\rm d}\xi \notag 
\end{align}
%
%
\begin{align}
    W'\left(\vstate{Y}{}{}+\Delta\vstate{Y}{}{}\right) &=2G\frac{h^3}{12}\frac{3\nu-1}{1-\nu}\left[\bar{\boldsymbol{\kappa}}\!\left(\vstate{Y}{}{}+\Delta\vstate{Y}{}{}\right)\right]^2 \notag \\
    &=2G\frac{h^3}{12}\frac{3\nu-1}{1-\nu} \left\{ \vphantom{\int_0^\delta} \bar{\boldsymbol{\kappa}}\!\left(\vstate{Y}{}{}\right) \cdot \bar{\boldsymbol{\kappa}}\!\left(\vstate{Y}{}{}\right)\right. \notag \\
    &\phantom{=2G\frac{h^3}{12}\frac{3\nu-1}{1-\nu}} + \frac{4}{m} \int_0^\delta \int_0^{2\pi}\omega(\xi)\frac{\Delta\vstate{Y}{}{\boldsymbol{\xi}}\cdot \bar{\boldsymbol{\kappa}}\!\left(\vstate{Y}{}{}\right)}{\xi^2} \xi {\rm d}\phi {\rm d}\xi \notag \\
    &\phantom{=2G\frac{h^3}{12}\frac{3\nu-1}{1-\nu}}\left. + \frac{2}{m} \int_0^\delta \int_0^{2\pi}\omega(\xi)\frac{\Delta\vstate{Y}{}{\boldsymbol{\xi}}\cdot \Delta\vstate{Y}{}{\boldsymbol{\xi}}}{\xi^2} \xi {\rm d}\phi {\rm d}\xi\right\} \notag \\
    &=W'\left(\vstate{Y}{}{}\right)+ 2G\frac{h^3}{12}\frac{3\nu-1}{1-\nu} \frac{4}{m} \frac{\omega(\xi)}{\xi^2} \bar{\boldsymbol{\kappa}}\!\left(\vstate{Y}{}{}\right) \bullet \Delta\vstate{Y}{}{} + o\left(||\Delta\vstate{Y}{}{}||\right)\notag \\
    \nabla W'\!\left(\vstate{Y}{}{}\right) &= \vstate{T}{}{\boldsymbol{\xi}} =  \frac{8 G}{m} \frac{h^3}{12}\frac{3\nu-1}{1-\nu}\frac{\omega(\xi)}{\xi^2} \bar{\boldsymbol{\kappa}}\notag
\end{align}
%
This demonstrates the bond-length dependent ``pressure'' applied to each point in the neighborhood of a point with average curvature $\bar{\boldsymbol{\kappa}}$.




\chapter{Notations }

Here we show the use of multiple appendixes.


\section{Math Notations}

Each appendix can have sub-sections as a regular chapter.