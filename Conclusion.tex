\chapter{Conclusion}

This work develops what we believe to be the first state-based peridynamic thin feature models.
These models allow for the simulation of bending in peridynamic beams, plates, and shells.
They are verified by comparing their strain energies to the strain energy of classical models for small homogenous deformations, which are matched for both Euler-Bernoulli beams and for Kirchhoff-Love plates as the peridynamic horizon approaches zero.
In addition to matching linear elastic models for these structures, it has been shown that the peridynamic models also match with Eringen-style gradient elasticity as the horizon decreases.

The bond-pair version of the bending model resists only bending deformation and is limited to plates with Poisson's ratios $\nu = \sfrac{1}{3}$.
To simulate loading incorporating forces both in-plane and transverse to the beam or plate, the pure bending model was combined with a conventional peridynamic model for in-plane deformation.
The result is a hybrid bending-extension model.
The Poisson's ratio limitation was overcome by decomposing bending deformation into isotropic and deviatoric components, allowing for a bond-multiple isotropic bending correction to the deformation energy.
With this correction, the deformation energy was shown to match that of a Kirchhoff-Love plate for any (valid) Poisson's ratio.

To further extend the utility of the model, a reformulation was developed to gracefully include curved and irregularly discretized areas.
By adding virtual points at locations where no nodes is present, the bond-pair model can be applied to regions where bonds are not initially organized into equal and opposite pairs.
This includes curved surfaces, where a virtual point may be added slightly above or below the surface, as well as irregularly discretized regions in which virtual points may be added between peridynamic nodes.

Code was written to evaluate both beam and plate models for a variety of support configurations, load configurations, and materials.
Although many problems of interest result in purely transverse deformation, both mathematical models and computer code are fully 3D.
The nature of peridynamic models results in large but easily parallelizable problems.
The code was written with this in mind, and scales easily to large problems on multiple machines.
Simulations run with the developed models provide results in agreement with conventional methods for several support configurations, including simply-supported, clamped, and free end conditions.
Good agreement was also reached for applied displacements, moments, and both point and distributed loading conditions.
The proposed damage model successfully reproduces the impact of nonlinear elasticity on deformation of a rectangular beam, and the framework is laid to allow application of the same model to a variety of beam cross sections.
Brittle fracture results for both beam and plate simulations are also consistent with expectation.
A single-torsion plate demonstrates the peridynamic advantages of natural crack path development without the need to predict direction of propagation.
The hybrid bending-extension model resists transverse, in-plane, and combined deformation, allowing it to successfully reproduce the stiffening effect of adding in-plane tension to a transversely-loaded plate.
With the bond-multiple correction, the extended model demonstrated the effect of varying Poisson's ratio on plate bending.
Because it was inspired by a similar decomposition used in extension models, there same procedure can be used for combined in-plane and transverse deformations.
Implementation of the virtual point extension greatly increases the practical utility of the model, a fact demonstrated by accurate simulations of the displacements of irregularly discretized plates and beams.
Finally, the addition of virtual points allows application to curved features, as demonstrated by successful simulation of beam and plate based rings.

The state-based nature of these models allows them to resist bending as a basic deformation mode, in a way that is fundamentally different from previous peridynamic models, both 3D solid and thin-feature.
While the examples in this work focus on conventional elastic and inelastic behavior, treating bending as a basic deformation opens the possibility of modeling thin features whose resistance to bending is not simply a function of stresses that vary through the thickness to produce a moment, and suggests a wide variety of possible applications: 
Graphene sheets are only one atom thick, so their bending behavior cannot be the result of through-thickness variation in stress. 
Some biological membranes resist isotropic bending but not deviatoric stresses.
Highly-structured metamaterials may also have bending properties that cannot be derived solely from bulk solid properties.

Peridynamic modeling is a technique that is swiftly growing in popularity, especially among those concerned with the propagation of material failure.
The bending behavior of thin features is a critical part of many engineering analyses that has been largely out of reach for reasonably-sized peridynamic models. 
These peridynamic bending models extend the domain of problems to which peridynamic modeling can be applied. \todo{Add a few paragraphs suggesting some future work extending this.}

\appendix
\chapter{Fr\'echet Derivative}
\label{sec:frechet}
\section{Definition}
The derivative of a function of a state is defined by Silling in \cite{silling2007peridynamic} as follows:
\begin{quote}
Let $\Psi$ be a function of a state, $\Psi(\cdot):\mathcal{A}_m\rightarrow\mathcal{L}_n$. Suppose there exists a state-valued function denoted $\nabla\Psi\in\mathcal{A}_{m+n}$ such that for any $\vstate{A}{}{}\in\mathcal{A}_m$ and any $\Delta\vstate{A}{}{}\in\mathcal{A}_m$,
\begin{equation}
  \Psi(\vstate{A}{}{}+\Delta\vstate{A}{}{})=\Psi(\vstate{A}{}{})+\nabla\Psi(\vstate{A}{}{})\bullet\Delta\vstate{A}{}{}+o(||\Delta\vstate{A}{}{}||).
\end{equation}
Then $\Psi$ is said to be \textit{differentiable} and $\nabla\Psi$ is called the \textit{Frechet derivative} of $\Psi$.
\end{quote}
This is a fairly straightforward way of defining a derivative with respect to a state.
Because the force vector-state and deformation vector-state are work conjugate, the force vector-state can be determined by taking the Fr\'echet derivative of energy with respect to the deformation vector-state.
\section{Bond-Pair Force}
For the bond-pair model, we derive the bond force function from the bond-pair energy function
%
\begin{align}
%
\tvstate{T}{}{\boldsymbol{\xi}} &= \nabla w\!\left(\tvstate{Y}{}{\boldsymbol{\xi}}\right)\notag \\
%
w&=\omega\!\left(\boldsymbol{\xi}\right)\alpha\left[1+\cos\!\left(\theta\!\left(\tvstate{Y}{}{\boldsymbol{\xi}},\tvstate{Y}{}{-\boldsymbol{\xi}}\right)\right)\right]\notag \\
%
w\!\left(\tvstate{Y}{}{\boldsymbol{\xi}}+\Delta\tvstate{Y}{}{\boldsymbol{\xi}}\right) &=
 \omega\!\left(\boldsymbol{\xi}\right)\alpha\left[1+\cos\!\left(\theta\!\left(\tvstate{Y}{}{\boldsymbol{\xi}}+\Delta\tvstate{Y}{}{\boldsymbol{\xi}},\tvstate{Y}{}{-\boldsymbol{\xi}}\right)\right)\right]\notag
\end{align}
%
\begin{multline}
\nabla w\!\left(\tvstate{Y}{}{\boldsymbol{\xi}}\right)\bullet\Delta\tvstate{Y}{}{\boldsymbol{\xi}}= 
w\!\left(\tvstate{Y}{}{\boldsymbol{\xi}}+\Delta\tvstate{Y}{}{\boldsymbol{\xi}}\right) -w\!\left(\tvstate{Y}{}{\boldsymbol{\xi}}\right) \notag\\
%
=\omega\!\left(\boldsymbol{\xi}\right)\alpha\sin\!\left(\theta\!\left(\tvstate{Y}{}{\boldsymbol{\xi}},\tvstate{Y}{}{-\boldsymbol{\xi}}\right)\right)\!\left[\theta\!\left(\tvstate{Y}{}{\boldsymbol{\xi}}+\Delta\tvstate{Y}{}{\boldsymbol{\xi}},\tvstate{Y}{}{-\boldsymbol{\xi}}\right)-\theta\!\left(\tvstate{Y}{}{\boldsymbol{\xi}},\tvstate{Y}{}{-\boldsymbol{\xi}}\right)\right]
\end{multline}
%
\begin{equation}
\left[\theta\!\left(\tvstate{Y}{}{\boldsymbol{\xi}}+\Delta\tvstate{Y}{}{\boldsymbol{\xi}},\tvstate{Y}{}{-\boldsymbol{\xi}}\right)-\theta\!\left(\tvstate{Y}{}{\boldsymbol{\xi}},\tvstate{Y}{}{-\boldsymbol{\xi}}\right)\right] = 
\frac{\Delta\tvstate{Y}{}{\boldsymbol{\xi}}}{|\tvstate{Y}{}{\boldsymbol{\xi}}|}\bullet \hat{\theta}\!\left(\tvstate{Y}{}{\boldsymbol{\xi}},\tvstate{Y}{}{-\boldsymbol{\xi}}\right)\notag
\end{equation}
%
To determine the $\hat{\theta}$ direction vector, we must construct a vector that is normal to $\tvstate{Y}{}{\boldsymbol{\xi}}$ and that is in the plane containing both $\vstate{Y}{}{\boldsymbol{\xi}}$ and $\vstate{Y}{}{\boldsymbol{-\xi}}$.
The cross product of $\vstate{Y}{}{\boldsymbol{\xi}}$ and $\vstate{Y}{}{\boldsymbol{-\xi}}$ is a vector normal to that plane, so any vector normal to that cross product will be in the correct plane.
Therefore, the vector $\vstate{Y}{}{\boldsymbol{\xi}}\times\left[\vstate{Y}{}{\boldsymbol{\xi}}\times\vstate{Y}{}{\boldsymbol{-\xi}}\right]$ is both normal to $\vstate{Y}{}{\boldsymbol{\xi}}$ and is in the plane containing both $\vstate{Y}{}{\boldsymbol{\xi}}$ and $\vstate{Y}{}{\boldsymbol{-\xi}}$.
Normalizing gives us the $\hat{\theta}$ direction vector:
%
\begin{equation}
\hat{\theta}\!\left(\tvstate{Y}{}{\boldsymbol{\xi}},\tvstate{Y}{}{-\boldsymbol{\xi}}\right)=
\frac{\tvstate{Y}{}{\boldsymbol{\xi}}\times\left[\tvstate{Y}{}{\boldsymbol{\xi}}\times\tvstate{Y}{}{\boldsymbol{-\xi}}\right]}{|\tvstate{Y}{}{\boldsymbol{\xi}}||\tvstate{Y}{}{\boldsymbol{\xi}}||\tvstate{Y}{}{\boldsymbol{-\xi}}|\sin\!\left(\theta\!\left(\tvstate{Y}{}{\boldsymbol{\xi}},\tvstate{Y}{}{-\boldsymbol{\xi}}\right)\right)}\notag 
\end{equation}
%
We combine all of these to get the expression for bond force found in \cref{eq:SillingForceNO}.
\begin{align}
%
\tvstate{T}{}{\boldsymbol{\xi}} &=
\omega\!\left(\boldsymbol{\xi}\right)\frac{-\alpha}{|\tvstate{Y}{}{\boldsymbol{\xi}}|} \frac{\tvstate{Y}{}{\boldsymbol{\xi}}}{|\tvstate{Y}{}{\boldsymbol{\xi}}|} \times 
\left[\frac{\tvstate{Y}{}{\boldsymbol{\xi}}}{|\tvstate{Y}{}{\boldsymbol{\xi}}|} \times 
\frac{\tvstate{Y}{}{-\boldsymbol{\xi}}}{|\tvstate{Y}{}{-\boldsymbol{\xi}}|}\right]\notag
\end{align}
%
\section{Isotropic Bending Correction}
To derive the bending ``pressure'' force, we start with the isotropic energy discrepancy
%
\begin{equation}
    W'=2\mu\frac{h^3}{12}\frac{3\nu-1}{1-\nu}\bar{\boldsymbol{\kappa}}^2. \notag
\end{equation}
%
with
%
\begin{align}
    \bar{\boldsymbol{\kappa}}\left(\vstate{Y}{}{}\right) &= \frac{1}{m} \int_0^\delta \int_0^{2\pi}\omega(\xi)\frac{\vstate{Y}{}{\boldsymbol{\xi}}+\vstate{Y}{}{\boldsymbol{-\xi}}}{\xi^2} \xi {\rm d}\phi {\rm d}\xi \notag \\
    &= \frac{2}{m} \int_0^\delta \int_0^{2\pi}\omega(\xi)\frac{\vstate{Y}{}{\boldsymbol{\xi}}}{\xi^2} \xi {\rm d}\phi {\rm d}\xi \notag
\end{align}
Because $\bar{\boldsymbol{\kappa}}$ is itself a vector-state, we will need to begin with the change in $\bar{\boldsymbol{\kappa}}$ with respect to $\vstate{Y}{}{}$ and carry the result through to find the change in $W'$.
\begin{align}
    \bar{\boldsymbol{\kappa}}\left(\vstate{Y}{}{}+\Delta\vstate{Y}{}{}\right) &= \frac{2}{m} \int_0^\delta \int_0^{2\pi}\omega(\xi)\frac{\vstate{Y}{}{\boldsymbol{\xi}}+\Delta\vstate{Y}{}{\boldsymbol{\xi}}}{\xi^2} \xi {\rm d}\phi {\rm d}\xi \notag \\
    &= \bar{\boldsymbol{\kappa}}\left(\vstate{Y}{}{}\right) + \frac{2}{m} \int_0^\delta \int_0^{2\pi}\omega(\xi)\frac{\Delta\vstate{Y}{}{\boldsymbol{\xi}}}{\xi^2} \xi {\rm d}\phi {\rm d}\xi \notag 
\end{align}
%
%
\begin{align}
    W'\left(\vstate{Y}{}{}+\Delta\vstate{Y}{}{}\right) &=2\mu\frac{h^3}{12}\frac{3\nu-1}{1-\nu}\left[\bar{\boldsymbol{\kappa}}\!\left(\vstate{Y}{}{}+\Delta\vstate{Y}{}{}\right)\right]^2 \notag \\
    &=2\mu\frac{h^3}{12}\frac{3\nu-1}{1-\nu} \left\{ \vphantom{\int_0^\delta} \bar{\boldsymbol{\kappa}}\!\left(\vstate{Y}{}{}\right) \cdot \bar{\boldsymbol{\kappa}}\!\left(\vstate{Y}{}{}\right)\right. \notag \\
    &\phantom{=2\mu\frac{h^3}{12}\frac{3\nu-1}{1-\nu}} + \frac{4}{m} \int_0^\delta \int_0^{2\pi}\omega(\xi)\frac{\Delta\vstate{Y}{}{\boldsymbol{\xi}}\cdot \bar{\boldsymbol{\kappa}}\!\left(\vstate{Y}{}{}\right)}{\xi^2} \xi {\rm d}\phi {\rm d}\xi \notag \\
    &\phantom{=2\mu\frac{h^3}{12}\frac{3\nu-1}{1-\nu}}\left. + \frac{2}{m} \int_0^\delta \int_0^{2\pi}\omega(\xi)\frac{\Delta\vstate{Y}{}{\boldsymbol{\xi}}\cdot \Delta\vstate{Y}{}{\boldsymbol{\xi}}}{\xi^2} \xi {\rm d}\phi {\rm d}\xi\right\} \notag \\
    &=W'\left(\vstate{Y}{}{}\right)+ 2\mu\frac{h^3}{12}\frac{3\nu-1}{1-\nu} \frac{4}{m} \frac{\omega(\xi)}{\xi^2} \bar{\boldsymbol{\kappa}}\!\left(\vstate{Y}{}{}\right) \bullet \Delta\vstate{Y}{}{} + o\left(||\Delta\vstate{Y}{}{}||\right)\notag \\
    \nabla W'\!\left(\vstate{Y}{}{}\right) &= \vstate{T}{}{\boldsymbol{\xi}} =  \frac{8 \mu}{m} \frac{h^3}{12}\frac{3\nu-1}{1-\nu}\frac{\omega(\xi)}{\xi^2} \bar{\boldsymbol{\kappa}}\notag
\end{align}
%
This demonstrates the bond-length dependent ``pressure'' applied to each point in the neighborhood of a point with average curvature $\bar{\boldsymbol{\kappa}}$.




\chapter{Notations }

Peridynamics is a new field, and different authors use a variety of notations to represent a variety of concepts.
Although chosen to be as consistent as possible with other authors, the following notation is therefore not portable to other works.

%%\begin{table}
%%\centering
%%\begin{tabular}{>{$\displaystyle}l<{$} r}
%\begin{longtable}{>{$\displaystyle}l<{$} r}
%\textrm{Notation} & Meaning \\ \hline\hline
%\hat{x},\hat{y},\hat{z} & coordinate vectors\\
%v & displacement in $\hat{y}$ direction\\
%w & displacement in $\hat{z}$ direction\\
%\mathbf{x},\mathbf{q} & undeformed location vectors \\
%\mathbf{u} & displacement vector \\
%\mathbf{y} & deformed location vector\\
%\boldsymbol{\xi}, \boldsymbol{\zeta} & undeformed bond vectors \\
%\nu & Poisson's ratio\\ 
%%K_I & mode I fracture toughness\\
%%\kappa & Kolosov constant\\
%%\theta & angle from crack tip\\
%%r & distance from crack tip\\
%\mu & shear modulus\\
%\rho &density\\
%\mathbf{P} & first Piola-Kirchhoff stress tensor\\
%\mathbf{b} & body force vector\\
%\Omega & undeformed body\\
%\mathbf{f} & force vector\\
%\boldsymbol{\sigma} & Cauchy stress tensor\\
%E & Young's modulus\\
%I & second moment of area\\
%q & distributed load (1D)\\
%M & moment\\
%V & shear force\\
%\rho & radius of curvature\\
%\mathcal{O} & order notation\\
%\theta & beam deflection angle\\
%\kappa & beam curvature\\
%A & area\\
%F & applied force\\
%\epsilon & uniaxial strain\\
%\boldsymbol{\epsilon} & strain tensor\\
%\kappa_x = \kappa_{xx} = \kappa_1 = \frac{\partial^2 w}{\partial x^2} & \multirow{3}{*}{Plate Curvatures} \\
%\kappa_y = \kappa_{yy} = \kappa_2 = \frac{\partial^2 w}{\partial y^2} & \\
%\kappa_{xy} = \kappa_3 = \frac{\partial^2 w}{\partial x\partial y} & \\
%D & plate flexural rigidity\\
%p & distributed load (2D)\\
%\vstate{T}{}{} & force vector state\\
%\vstate{Y}{}{} & deformation vector state\\
%\omega & weight function\\
%\delta & horizon\\
%\mathcal{H} & neighborhood\\
%c & bond stiffness\\
%s & bond stretch\\
%\mathbf{F} & deformation gradient tensor\\
%\xi_i & $i$-th component of vector $\boldsymbol{\xi}$\\
%\xi & magnitude of vector $\boldsymbol{\xi}$\\
%\boldsymbol{\xi}_i & $i$-th of $n$ vectors vector $\boldsymbol{\xi}_1$ to $\boldsymbol{\xi}_n$  \\
%\mathbf{p},\mathbf{q},\mathbf{r} & deformed bond vectors\\
%w & bond energy, bond-pair energy\\
%W & strain or deformation energy density\\
%\epsilon^d & deviatoric strain\\
%m & peridynamic normalization constant\\
%\theta & dilation\\
%t & thickness\\
%\underline{e} & bond stretch\\
%\underline{e}^i &isotropic portion of bond stretch\\
%\underline{e}^d &deviatoric portion of bond stretch\\
%\mathbf{K} & shape tensor\\
%\Omega & classical strain energy density function\\
%\alpha & peridynamic bending stiffness\\
%\theta & bond-pair angle\\
%\bar{\kappa} & nonlocal curvature\\
%K & Eringen's nonlocal modulus\\
%\mathbf{t} & Eringen nonlocal stress\\
%\tau^2l^2 & scale parameter\\
%t_{1D} & 1D Eringen nonlocal stress\\
%\sigma_{1D} & 1D local stress\\
%b & beam width\\
%\epsilon_c & critical strain\\
%\theta_c & critical angle\\
%\theta_2 & 2D dilation\\
%\bar{\kappa} & scalar isotropic curvature\\
%\bar{\boldsymbol{\kappa}} & vector isotropic curvature\\
%W' & energy discrepancy\\
%\vstate{T'}{}{} & correction force vector state\\
%\alpha^{\textrm{iso}} & isotropic bending stiffness\\
%f^\textrm{iso} & isotropic bending ``pressure''\\
%\hline\hline
%%\end{tabular}
%\caption{Peridynamic Notation}
%\label{table:PDnotation}
%\end{longtable}
%%\end{table}


\begin{longtable}{>{$\displaystyle}l<{$} r}
\textrm{Notation} & Meaning \\ \hline\hline
A & area\\
b & beam width\\
\mathbf{b} & body force vector\\
c & bond stiffness\\
D & plate flexural rigidity\\
E & Young's modulus\\
\underline{e} & bond stretch\\
\underline{e}^i &isotropic portion of bond stretch\\
\underline{e}^d &deviatoric portion of bond stretch\\
F & applied force\\
\mathbf{f} & force vector\\
\mathbf{F} & deformation gradient tensor\\
f^\textrm{iso} & isotropic bending ``pressure''\\
\mathcal{H} & neighborhood\\
I & second moment of area\\
K & Eringen's nonlocal modulus\\
\mathbf{K} & shape tensor\\
M & moment\\
m & peridynamic normalization constant\\
\mathcal{O} & order notation\\
p & distributed pressure load (2D)\\
\mathbf{P} & first Piola-Kirchhoff stress tensor\\
\mathbf{p},\mathbf{q},\mathbf{r} & deformed bond vectors\\
q & distributed load (1D)\\
s & bond stretch\\
t & thickness\\
\mathbf{t} & Eringen nonlocal stress\\
t_{1D} & 1D Eringen nonlocal stress\\
\vstate{T}{}{} & force vector state\\
\vstate{T'}{}{} & correction force vector state\\
\mathbf{u} & displacement vector \\
V & shear force\\
v & displacement in $\hat{y}$ direction\\
w & displacement in $\hat{z}$ direction\\
w & bond energy, bond-pair energy\\
W & strain or deformation energy density\\
W' & energy discrepancy\\
\hat{x},\hat{y},\hat{z} & coordinate vectors\\
\mathbf{x},\mathbf{q} & undeformed location vectors \\
\mathbf{y} & deformed location vector\\
\vstate{Y}{}{} & deformation vector state\\


\alpha & peridynamic bending stiffness\\
\alpha^{\textrm{iso}} & isotropic bending stiffness\\
\delta & horizon\\
\epsilon & uniaxial strain\\
\boldsymbol{\epsilon} & strain tensor\\
\epsilon_c & critical strain\\
\epsilon^d & deviatoric strain\\
\theta & beam deflection angle\\
\theta & dilation\\
\theta & bond-pair angle\\
\theta_c & critical angle\\
\theta_{2D} & 2D dilation\\
\kappa & beam curvature\\
\bar{\kappa} & nonlocal curvature\\
\bar{\boldsymbol{\kappa}} & vector isotropic curvature\\
\kappa_x = \kappa_{xx} = \kappa_1 = \frac{\partial^2 w}{\partial x^2} & \multirow{3}{*}{Plate Curvatures} \\
\kappa_y = \kappa_{yy} = \kappa_2 = \frac{\partial^2 w}{\partial y^2} & \\
\kappa_{xy} = \kappa_3 = \frac{\partial^2 w}{\partial x\partial y} & \\
\mu & shear modulus\\
\nu & Poisson's ratio\\ 
\boldsymbol{\xi}, \boldsymbol{\zeta} & undeformed bond vectors \\
\boldsymbol{\xi}_i & $i$-th of $n$ vectors $\boldsymbol{\xi}_1$ to $\boldsymbol{\xi}_n$  \\
\xi_i & $i$-th component of vector $\boldsymbol{\xi}$\\
\xi & magnitude of vector $\boldsymbol{\xi}$\\
\rho &density\\
\rho & radius of curvature\\
\sigma_{1D} & 1D local stress\\
\boldsymbol{\sigma} & Cauchy stress tensor\\
\tau^2l^2 & scale parameter\\
\Omega & undeformed body\\
\Omega & classical strain energy density function\\
\omega & weight function\\

\hline\hline
%\end{tabular}
\caption{Peridynamic Notation}
\label{table:PDnotation}
\end{longtable}


