\committee{John T. Foster, Ph.D., Chair}{Yusheng Feng, Ph.D.}{Walter Richardson, Ph.D.}{James Walker, Ph.D. }{Xiaowei Zeng, Ph.D. }


\informationitems{Doctor of Philosophy in Mechanical Engineering}{Ph.D.}{M.S.M.E.}{Department of Mechanical Engineering}{College of Engineering}{September}{ 2014 }


\thesiscopyright{Copyright 2014 James O'Grady \\
All rights reserved. }


\dedication{\emph{I would like to dedicate this dissertation to the memory of 
Thomas John O'Grady II, a passionate father and a passionate engineer.}}


\title{\textbf{PERIDYNAMIC BEAMS, PLATES, AND SHELLS:
}\\
\textbf{A NONORDINARY, STATE-BASED MODEL}}


\author{James O'Grady}
\maketitle
\begin{acknowledgements}
This work was funded by grant number W911NF-11-1-0208 from the United States Air Force Office of Scientific Research. 

I would like to thank my advisor, Dr. John Foster, for steering me onto a road I never knew existed and for pointing me at such an excellent problem.

Most of all, I would like to thank my wife, who has been endlessly supportive.


\begin{singlespace}
\emph{This Masters Thesis/Recital Document or Doctoral Dissertation
was produced in accordance with guidelines which permit the inclusion
as part of the Masters Thesis/Recital Document or Doctoral Dissertation
the text of an original paper, or papers, submitted for publication.
The Masters Thesis/Recital Document or Doctoral Dissertation must
still conform to all other requirements explained in the Guide for
the Preparation of a Masters Thesis/Recital Document or Doctoral Dissertation
at The University of Texas at San Antonio. It must include a comprehensive
abstract, a full introduction and literature review, and a final overall
conclusion. Additional material (procedural and design data as well
as descriptions of equipment) must be provided in sufficient detail
to allow a clear and precise judgment to be made of the importance
and originality of the research reported. }

\emph{It is acceptable for this Masters Thesis/Recital Document or
Doctoral Dissertation to include as chapters authentic copies of papers
already published, provided these meet type size, margin, and legibility
requirements. In such cases, connecting texts, which provide logical
bridges between different manuscripts, are mandatory. Where the student
is not the sole author of a manuscript, the student is required to
make an explicit statement in the introductory material to that manuscript
describing the students contribution to the work and acknowledging
the contribution of the other author(s). The signatures of the Supervising
Committee which precede all other material in the Masters Thesis/Recital
Document or Doctoral Dissertation attest to the accuracy of this statement.}\end{singlespace}
\end{acknowledgements}
\begin{abstract}
Peridynamics is a nonlocal formulation of continuum mechanics in which forces are calculated as integral functions of displacement fields rather than spatial derivatives.
The peridynamic model has major advantages over classical continuum mechanics when displacements are discontinuous, such as in the case of material failure.
While multiple peridynamic material models capture the behavior of solid materials, not all structures are conveniently analyzed as solids.
Finite Element Analysis often uses 1D and 2D elements to model thin features that would otherwise require a great number of 3D elements, but peridynamic thin features remain underdeveloped despite great interest in the engineering community.
This work develops nonordinary state-based peridynamic models for the simulation of thin features.
Beginning from an example nonordinary state-based model, lower dimensional peridynamic models of plates, beams, and shells are developed and validated against classical models.
These peridynamic models are extended to incorporate brittle and plastic material failure, compounding the peridynamic advantages of discontinuity handling with the computational simplicity of reduced-dimension features.
Once validated\todo{Aren't you validating them as part of this work?  I would remove ``Once validated''}, these models will allow peridynamic modeling of complex structures such as aircraft skin that may experience damage from internal forces or external impacts.
\end{abstract}

\pageone{}
